% =====================================================================
% IEEE GRSL - Environmental Detectability of Refugee Camps
% Author: Jose Miguel Robles Leyton
% Target: IEEE Geoscience and Remote Sensing Letters (5 pages max)
% =====================================================================

\documentclass[journal]{IEEEtran}

\usepackage{cite}
\usepackage{amsmath,amssymb,amsfonts}
\usepackage{graphicx}
\usepackage{textcomp}
\usepackage{xcolor}
\usepackage{booktabs}
\usepackage{url}

\hyphenation{Senti-nel}

\begin{document}

\title{NDBI Spectral Gap Predicts Refugee Camp Detectability from Sentinel-2: Environmental Limits at 10\,m Resolution}

\author{Jose~Miguel~Robles~Leyton%
\thanks{J. M. Robles Leyton is an Independent Researcher, Utah, USA
(e-mail: jose.m.robles7@gmail.com).}%
\thanks{Manuscript received XX, 2026.}}

\maketitle

% =====================================================================
% ABSTRACT — max 250 words
% =====================================================================
\begin{abstract}
Detecting refugee camps from medium-resolution satellite imagery is critical
for humanitarian response, yet cross-country generalization remains poorly
understood. We investigate when refugee camps are spectrally detectable using
Sentinel-2 at 10\,m resolution. Using 101 camps across seven countries and
1689 image tiles, we evaluate a leave-one-country-out spectral classifier
based on the Normalized Difference Built-Up Index. Detection performance
varies dramatically by country, with area under the ROC curve ranging from
0.24 to 0.77. This variance is explained by a single physical variable: the
NDBI spectral gap between camp and background. Country-level Pearson
correlation is 0.912, explaining 83\% of the variance, confirmed by
permutation testing and leave-two-countries-out validation. Camps built with
concrete or metal on rural land are detectable, while camps of mud and thatch
on spectrally similar soil are invisible. A convolutional neural network
trained cross-country performs below chance, confirming a sensor-physics
limitation rather than an algorithmic one. Structural texture features
partially recover signal where spectral contrast is absent, but their
discriminative direction varies by environment, precluding any universal
settlement signature at this resolution.
\end{abstract}

\begin{IEEEkeywords}
Remote sensing, Sentinel-2, refugee settlements, spectral analysis,
built-up index, image classification, humanitarian applications
\end{IEEEkeywords}

\IEEEpeerreviewmaketitle

% =====================================================================
% I. INTRODUCTION
% =====================================================================
\section{Introduction}

\IEEEPARstart{S}{atellite-based} detection of refugee and internally
displaced persons (IDP) camps is essential for humanitarian operations when
ground access is restricted. Very high resolution (VHR, $<$1\,m) imagery
enables dwelling-level extraction~\cite{Quinn2018, Gella2022}, but coverage
is limited and commercially licensed. Sentinel-2, a freely available
ESA mission with global coverage at 10\,m and a 5-day revisit, offers a
scalable alternative, yet only two studies have tested it for refugee
camps~\cite{Wendt2017, Wernicke2023}, with no cross-country analysis.

Global settlement products derived from Sentinel-2 fail catastrophically on
refugee camps. Van Den Hoek and Friedrich~\cite{VanDenHoek2021} showed that
GHS-BUILT-S2 achieves a median F1 of only 0.15 at 30~refugee settlements in
Uganda, while the World Settlement Footprint missed 8 of 30 entirely.
Pesaresi~et~al.~\cite{Pesaresi2013} acknowledged that ``scattered huts in
rural areas built with traditional materials such as straw or clay'' are
``not distinguishable from background.'' However, no prior work has
quantified \emph{when} or \emph{why} detection fails.

We address a different question than detection: \emph{under what
environmental conditions is spectral detection physically possible?} We show
that the NDBI spectral gap ($\Delta$NDBI $=$ mean NDBI of camp $-$ mean NDBI
of background) explains 83\% of cross-country detection variance. This
transforms scattered qualitative observations~\cite{Kuffer2016, Rasul2018}
into a quantitative, predictive framework. Our contribution is the
\emph{characterization} of detectability limits, not a new detector.

% =====================================================================
% II. DATA AND METHODS
% =====================================================================
\section{Data and Methods}

\subsection{Dataset}

We compiled 101~refugee/IDP camps across seven countries---selected to
span arid (Chad, Yemen), semi-arid (Turkey, Syria, Ethiopia), and humid
(Uganda, South Sudan) biomes---from UNHCR/HDX databases and
OpenStreetMap (Table~\ref{tab:loco}). For each camp, we downloaded
Sentinel-2 L2A imagery (2022--2023, cloud $<$20\%) from Microsoft
Planetary Computer. We extracted six spectral bands (B02--B04, B08, B11,
B12) and derived three indices (NDVI, NDBI, SWIR ratio), producing
128$\times$128 pixel tiles at 10\,m (1.28\,km$\times$1.28\,km). SWIR bands
(B11, B12; native 20\,m) were resampled to 10\,m with bilinear
interpolation. Each camp has five tiles (center~$+$~four adjacent); negative
samples (rural, urban, barren, informal settlements) were placed $>$10\,km
from any camp. The final dataset comprises 1689~tiles (505~camp,
1184~negative).

\subsection{Spectral Model}

We trained a logistic regression classifier on three NDBI-derived features
(mean, standard deviation, and fraction of pixels with NDBI~$>$~0) with
class-weight balancing. Evaluation follows leave-one-country-out
cross-validation (LOCO): for each of seven countries, we train on the
remaining six and test on the held-out country, reporting
area-under-the-ROC-curve (AUC).

\subsection{NDBI Spectral Gap}

For each country, we define the spectral gap as
\begin{equation}
\Delta\text{NDBI} = \overline{\text{NDBI}}_{\text{camp}} -
\overline{\text{NDBI}}_{\text{neg}}
\label{eq:gap}
\end{equation}
where averages are taken over all tiles in that country. This measures
whether camp building materials produce a spectral contrast against the
natural background.

\subsection{Structural Texture Features}

To test whether spatial structure compensates for spectral invisibility, we
computed 17~texture features from the Red band (B04, native 10\,m)
following~\cite{Wurm2017}: gradient
magnitude (Sobel), edge density, directional entropy, GLCM properties
(homogeneity, contrast, correlation, dissimilarity, energy; distance$=$2,
four angles, 16~gray levels), and nine FFT radial power-spectrum bins (17~features total). A separate
logistic regression model was trained on these features under the same LOCO
protocol.

\subsection{CNN Baseline}

As a morphological baseline, we trained a ResNet-18 on all nine
channels (six bands $+$ three indices) using four countries (Chad,
Ethiopia, Syria, Uganda) and tested on the remaining three (Turkey,
Yemen, South Sudan), with 50~epochs, Adam optimizer, and batch size~32.

\subsection{Statistical Validation}

We assess the $\Delta$NDBI--AUC correlation using: (1) Pearson $r$ at
country level ($n=7$); (2) leave-two-countries-out (L2CO, $n=42$
evaluations); (3) bootstrap confidence interval (10,000 resamples); and (4)
exact permutation test ($7!=5040$ permutations).

% =====================================================================
% III. RESULTS
% =====================================================================
\section{Results}

\subsection{Spectral Detectability is Environment-Dependent}

LOCO AUC varies from 0.238 (South Sudan) to 0.773 (Turkey), with mean
0.598$\pm$0.174 (Table~\ref{tab:loco}). Country-level correlation between
$\Delta$NDBI~(\ref{eq:gap}) and LOCO AUC yields $r=+0.912$ ($R^2=0.832$;
Fig.~\ref{fig:main}a). The permutation test confirms significance
($p=0.005$; only 25 of 5040 permutations produce $|r|\geq0.912$). Bootstrap
95\% CI is $[+0.56, +1.00]$, though we note under-coverage at $n=7$. L2CO
validation ($n=42$) yields $r=+0.872$ ($R^2=0.760$), with per-country AUC
stability: South Sudan consistently undetectable
($0.235\pm0.005$), Turkey consistently detectable ($0.750\pm0.055$).

\begin{table}[t]
\centering
\caption{LOCO results: spectral and structural models}
\label{tab:loco}
\begin{tabular}{@{}lrccc@{}}
\toprule
Country & Camps & $\Delta$NDBI & \multicolumn{2}{c}{LOCO AUC} \\
\cmidrule(l){4-5}
 & & & Spectral & Texture \\
\midrule
Turkey      & 6  & $+$0.129 & 0.773 & 0.629 \\
Uganda      & 6  & $+$0.055 & 0.654 & 0.634 \\
Chad        & 19 & $+$0.046 & 0.652 & 0.536 \\
Ethiopia    & 26 & $+$0.023 & 0.678 & 0.611 \\
Yemen       & 6  & $+$0.020 & 0.658 & 0.584 \\
Syria       & 28 & $-$0.001 & 0.531 & 0.426 \\
S.~Sudan    & 10 & $-$0.077 & 0.238 & \textbf{0.542} \\
\midrule
Mean (Total) & 101 & & 0.598 & 0.566 \\
\bottomrule
\end{tabular}
\end{table}

\begin{figure*}[t]
\centering
\includegraphics[width=\textwidth]{roble1}
\caption{Environmental determinants of refugee camp detectability.
(a)~NDBI spectral gap versus LOCO AUC at country level ($r=+0.912$,
$R^2=0.832$, permutation $p=0.005$). Each point is one country.
(b)~ROC curves for South Sudan comparing spectral (AUC$=$0.24) and
structural texture (AUC$=$0.54) pathways.
(c)~Cohen's $d$ for GLCM homogeneity by country. The structural
signal direction varies by environment: South Sudan camps are smoother
than surroundings ($d>0$), while arid-country camps are more
textured ($d<0$), illustrating the biome-dependent inversion.}
\label{fig:main}
\end{figure*}

\subsection{Physical Mechanism}

Turkey ($\Delta$NDBI$=+0.129$): concrete container camps on rural land
create strong spectral contrast. South Sudan
($\Delta$NDBI$=-0.077$): mud-and-thatch camps have \emph{lower} NDBI than
surrounding vegetation---the camp is spectrally \emph{invisible}. Syria
($\Delta$NDBI$\approx0$): camps in semi-urbanized areas are indistinguishable
from background. A ResNet-18 CNN trained on four countries and tested on
three unseen countries achieves test AUC$=$0.279 (versus 0.728 for NDBI
logistic regression on the same split), confirming the signal is
physical, not morphological, at 10\,m.

\subsection{Structural Features Partially Recover Signal}

In South Sudan, texture features improve AUC from 0.238 to 0.542
(Fig.~\ref{fig:main}b). Camps are significantly smoother and more
homogeneous than surroundings (GLCM homogeneity Cohen's $d=+0.68$,
$p<0.001$; gradient magnitude $d=-0.66$, $p<0.001$), reflecting cleared
ground amid vegetation.

However, the \emph{direction} of textural discrimination varies by
environment (Fig.~\ref{fig:main}c): in arid countries, camps tend to be
more textured than bare surroundings (structures on bare ground), while
in South Sudan, camps are smoother (cleared ground in vegetation). A global
texture model therefore performs worse than the spectral model
(0.566 vs.\ 0.598). An oracle selector (best pathway per country) reaches
0.641, but learning the switching threshold from training countries fails
($n=7$ is insufficient, and South Sudan is the only country with negative
gap).

% =====================================================================
% IV. DISCUSSION
% =====================================================================
\section{Discussion}

Prior work has documented individual pieces of this puzzle: NDBI limitations
in arid zones~\cite{Zha2003, Rasul2018}, sub-pixel detection limits for
Sentinel-2~\cite{Radoux2016}, settlement product failures on refugee
camps~\cite{VanDenHoek2021}, and CNN generalization
challenges~\cite{Gella2022, Corbane2021}. No prior study has connected these
into a unified physical framework: sensor physics $\to$ spectral contrast
$\to$ statistical separability $\to$ ML performance.

The $\Delta$NDBI metric transforms Pesaresi et al.'s qualitative
observation~\cite{Pesaresi2013} into a quantitative predictor: when
$\Delta$NDBI$\leq0$, spectral detection is not viable regardless of
classifier complexity. This explains why global products
fail~\cite{VanDenHoek2021}: they are validated on cities where spectral
contrast is high, not on informal settlements where it can vanish.

The texture signal inversion between biomes is, to our knowledge,
unreported. It implies that no universal camp signature exists at 10\,m
resolution---neither spectral nor structural. Any detection pipeline must
incorporate environmental characterization \emph{a priori}.

\textbf{Limitations:} Our study includes seven countries ($n=7$ for
country-level analysis), mitigated by L2CO ($n=42$) and per-camp ($n=101$)
validations. SWIR bands are resampled from 20\,m, and our texture analysis
uses the 10\,m Red band. Future work should explore Sentinel-1 SAR~\cite{Braun2019}, whose
backscatter responds to surface roughness rather than reflectance and could
recover geometric signatures in spectrally invisible environments.

% =====================================================================
% V. CONCLUSION
% =====================================================================
\section{Conclusion}

Refugee camp detectability from Sentinel-2 at 10\,m is contingent on
environmental context. The NDBI spectral gap---a simple, interpretable
metric---explains 83\% of cross-country detection variance and can serve as
a pre-screening tool before any classification effort. No universal camp
signature exists at this resolution: spectral features fail when building
materials match background, and structural features invert direction between
biomes. These are sensor-physics constraints, not algorithmic ones.

\section*{Data Availability}
Sentinel-2 imagery was accessed via the Microsoft Planetary Computer.
Camp location data were compiled from publicly available humanitarian
databases (UNHCR, HDX, OpenStreetMap). Derived features and analysis
code are available from the author upon reasonable request.

\section*{Acknowledgment}

The author transparently discloses that this work was produced
through extensive coordination of two AI systems: ChatGPT 5.2 (OpenAI)
for research conception, experimental design, and scientific
interpretation, and Claude Code with Opus 4.6 (Anthropic) for data processing,
statistical modeling, code implementation, figure generation,
literature review, and manuscript preparation. AI-generated content
is present in all sections of this article, including text, figures,
and code. Both systems were used iteratively in combination to obtain
verifiable and reproducible results; all scientific claims were
cross-verified against published peer-reviewed sources. Sentinel-2 data were accessed via the
Microsoft Planetary Computer. The author assumes full responsibility
for the accuracy and integrity of this work.

% =====================================================================
% REFERENCES — IEEE numbered style, verified DOIs
% =====================================================================
\begin{thebibliography}{13}

\bibitem{Quinn2018}
J.~A.~Quinn \emph{et~al.}, ``Humanitarian applications of machine learning
with remote-sensing data: Review and case study in refugee settlement
mapping,'' \emph{Phil. Trans. R. Soc. A}, vol.~376, no.~2128, Art.~no.
20170363, 2018, doi:~10.1098/rsta.2017.0363.

\bibitem{Gella2022}
G.~W.~Gella \emph{et~al.}, ``Mapping of dwellings in IDP/refugee settlements
from very high-resolution satellite imagery using a mask region-based
convolutional neural network,'' \emph{Remote Sens.}, vol.~14, no.~3,
Art.~no.~689, 2022, doi:~10.3390/rs14030689.

\bibitem{Wendt2017}
L.~Wendt, S.~Lang, and E.~Rogenhofer, ``Monitoring of refugee and camps for
internally displaced persons using Sentinel-2 imagery---A feasibility
study,'' \emph{GI\_Forum}, vol.~1, pp.~172--182, 2017,
doi:~10.1553/giscience2017\_01\_s172.

\bibitem{Wernicke2023}
K.~Wernicke, ``Deep learning for refugee camps---Mapping settlement extents
with Sentinel-2 imagery and semantic segmentation,'' M.S.~thesis, Dept.
Remote Sens., Univ.~W\"{u}rzburg, W\"{u}rzburg, Germany, 2023. [Online]. Available:
\url{https://elib.dlr.de/196349/}.

\bibitem{VanDenHoek2021}
J.~Van Den Hoek and H.~K.~Friedrich, ``Satellite-based human settlement
datasets inadequately detect refugee settlements: A critical assessment at
thirty refugee settlements in Uganda,'' \emph{Remote Sens.}, vol.~13,
no.~18, Art.~no.~3574, 2021, doi:~10.3390/rs13183574.

\bibitem{Pesaresi2013}
M.~Pesaresi \emph{et~al.}, ``A global human settlement layer from optical
HR/VHR RS data: Concept and first results,'' \emph{IEEE J. Sel. Topics Appl.
Earth Observ. Remote Sens.}, vol.~6, no.~5, pp.~2102--2131, Oct.~2013,
doi:~10.1109/JSTARS.2013.2271445.

\bibitem{Kuffer2016}
M.~Kuffer, K.~Pfeffer, and R.~Sliuzas, ``Slums from space---15 years of
slum mapping using remote sensing,'' \emph{Remote Sens.}, vol.~8, no.~6,
Art.~no.~455, 2016, doi:~10.3390/rs8060455.

\bibitem{Rasul2018}
A.~Rasul \emph{et~al.}, ``Applying built-up and bare-soil indices from
Landsat~8 to cities in dry climates,'' \emph{Land}, vol.~7, no.~3,
Art.~no.~81, 2018, doi:~10.3390/land7030081.

\bibitem{Wurm2017}
M.~Wurm, M.~Weigand, A.~Schmitt, C.~Geiss, and H.~Taubenb\"{o}ck,
``Exploitation of textural and morphological image features in Sentinel-2A
data for slum mapping,'' in \emph{Proc. Joint Urban Remote Sens. Event
(JURSE)}, Dubai, UAE, 2017, pp.~1--4, doi:~10.1109/JURSE.2017.7924586.

\bibitem{Zha2003}
Y.~Zha, J.~Gao, and S.~Ni, ``Use of normalized difference built-up index
in automatically mapping urban areas from TM imagery,'' \emph{Int. J. Remote
Sens.}, vol.~24, no.~3, pp.~583--594, 2003, doi:~10.1080/01431160304987.

\bibitem{Radoux2016}
J.~Radoux \emph{et~al.}, ``Sentinel-2's potential for sub-pixel landscape
feature detection,'' \emph{Remote Sens.}, vol.~8, no.~6, Art.~no.~488, 2016,
doi:~10.3390/rs8060488.

\bibitem{Corbane2021}
C.~Corbane \emph{et~al.}, ``Convolutional neural networks for global human
settlements mapping from Sentinel-2 satellite imagery,'' \emph{Neural
Comput. \& Appl.}, vol.~33, no.~12, pp.~6697--6720, 2021,
doi:~10.1007/s00521-020-05449-7.

\bibitem{Braun2019}
A.~Braun, F.~Fakhri, and V.~Hochschild, ``Refugee camp monitoring and
environmental change assessment of Kutupalong, Bangladesh, based on radar
imagery of Sentinel-1 and ALOS-2,'' \emph{Remote Sens.}, vol.~11, no.~17,
Art.~no.~2047, 2019, doi:~10.3390/rs11172047.

\end{thebibliography}

\end{document}
